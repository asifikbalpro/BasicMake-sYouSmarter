\newpage
\begin{document}
	A series expansion is a representation of a particular function as a sum of powers in one of its variables, or by a sum of powers of another (usually elementary) function  f(x).
	
	Here are series expansions (some Maclaurin, some Laurent, and some Puiseux) for a number of common functions.
%cannot have empty equation.
%and \\ is used for new line like '\n' in programming
%and \\ is also used for new equation in aligh
%got it
%and % is used for comment
		\begin{align}
		\frac{1}{1-x}=1+x+x^{2}+x^{3}+x^{4}+x^{5}+........for -1<x<1 \\
		cn(x,k)=1-\frac{1}{2}x^{2}+\frac{1}{24}(1+4k^{2})x^{4}+.........\\
		\end{align}
		\begin{align}
		\cos x &= 1- \frac{1}{2}x^{2}+\frac{1}{24}x^{4}-\frac{1}{720}x^{6} - \ldots for -\infty < x < \infty	\\
		\cos^{-1} x &= \frac{1}{2} \pi - x - \frac{1}{6} x^{3} - \frac{3}{40} x^{5} - \frac{5}{112} x^{7} - ......for - 1 < x < 1\\
		\cosh x &= 1 + \frac{1}{2} x^2  + \frac{1}{24} x^4 + \frac{1}{720} x^6 = \frac{1}{40320} x^8 + ....\\	
		\cosh^{-1} (1+x) &= \sqrt{2x} (1 - \frac{1}{12} x + \frac{3}{160} ^2 - \frac{5}{896} x^3 + ......) \\
		\cot x &= x^{-1} - \frac{1}{3} x - \frac{1}{45} x^3 - \frac{2}{945} x^5 - \frac{1}{4725} x^7 - .....\\
		\cot^{-1} x &= \frac{1}{2} \pi - x + \frac{1}{3} x^3 - \frac{1}{5} x^5 + \frac{1}{7} x^7 - \frac{1}{9} x^9 + ....\\	
		\cot_{-1}(\frac{1}{x}) &= x - \frac{1}{3} x^3 + \frac{1}{5} x x^5 - \frac{1}{7} x^7 + \frac{1}{9} x^9 + ....\\
		\cosh x &= x^{-1} + \frac{1}{3} x - \frac{1}{45} x^3 + \frac{2}{945} x^5 - \frac{1}{4725} x^7 + ....\\
		\coth^{-1}(1+x) &= \frac{1}{2} \ln 2 - \frac{1}{2} \ln x + \frac{1}{4} x - \frac{1}{16} x^2 + ....\\
		\csc x &= x^{-1} + \frac{1}{6} x + \frac{7}{360} x^3 + \frac{31}{15120} x^5 + ....\\
		%problem csc  is csch
		\csc x &= x^{-1} - \frac{1}{6} x + \frac{7}{360} x^3 + \frac{31}{15120} x^5 + ....\\
		\csc^{-1} x &= \ln 2 - \ln x + \frac{1}{4} x^2 - \frac{3}{32} x^4 + \frac{5}{96} x^6 - ....\\
		%problem ... 
		dn(x,k) &= 1 - \frac{1}{2} k^2 x^2 + \frac{1}{24} k^2 (4+k^2) x^4 + ....\\
		%probelm ...
		erf x &= \frac{1}{\sqrt{\pi}} (2x - \frac{2}{3} x^3 + \frac{1}{5} x^5 - \frac{1}{21}x^7 + ...) \\
		%probelm....
		e^x &= 1 + x + \frac{1}{2} x^2 + \frac{1}{6} x^3 + \frac{1}{24} x^4 + ... for - \infty < x < \infty \\
		_{2}F_{1} (\alpha,\beta,\gamma,x) &= 1 + \frac{\alpha\beta}{1!\gamma} x + \frac{\alpha(\alpha + 1)\beta(\beta + 1)}{2!\gamma(\gamma + 1)} x^2 ....\\
		\ln(1 + x) &= x - \frac{1}{2} x^2 + \frac{1}{3} x^3 - \frac{1}{4} x^4 + ....for -1 < x < 1 \\
		\ln (\frac{1+x}{1-x}) &= 2x + \frac{2}{3} x^3 + \frac{2}{5} x^5 + \frac{2}{7} x^7 + ..... for -1 < x < 1 \\
		\sec x &= 1 + \frac{1}{2} x^2 + \frac{5}{24} x^4 + \frac{61}{720} x^6 + \frac{277}{8064} x^8 + ....\\
		% problem sec is sech
		\sec x = 1 - \frac{1}{2} x^2 + \frac{5}{24} x^4 - \frac{61}{720} x^6 + \frac{277}{8064} x^8 + ....\\
 		\end{align}
		\begin{align}
		%problem sec is sech
		\sec^{-1} x &= \ln 2 - \ln x - \frac{1}{4} x^2 - \frac{3}{32} x^4 - .....\\
		\sin x &= x - \frac{1}{6} x^3 + \frac{1}{120} x^5 - \frac{1}{5040} x^7 ... for - \infty < x < \infty \\
		\sin^{-1} x &= x + \frac{1}{6} x^3 + \frac{3}{40} x^5 + \frac{5}{112} x^7 + \frac{35}{1152} x^9 + ....\\
		\sinh x &= x + \frac{1}{6} x^3 + \frac{1}{120} x^5 + \frac{1}{5040} x^7 + \frac{1}{362880} x^9 + ....\\
		\sinh^{-1} x &= x - \frac{1}{6} x^3 + \frac{3}{40} x^5 - \frac{5}{112} x^7 + \frac{35}{1152} x^9 -  ... \\
		sn(x,k) &= x - \frac{1}{6}(1+k^2) x^3 + \frac{1}{120} (1 + 14k^2 + k^4) x^5 + .... \\
		\tan x &= x + \frac{1}{3} x^3 + \frac{2}{15} x^5 + \frac{17}{315} x^7 + \frac{62}{2835} x^9 + ....\\
		\tan^{-1} x &= x - \frac{1}{3} x^3 + \frac{1}{5} x^5 - \frac{1}{7} x^7 + ..... for -1 < x < 1 \\
		\tan^{-1}(1+x) &= \frac{1}{4} \pi + \frac{1}{2} x - \frac{1}{4} x^2 + \frac{1}{12} x ^3 + \frac{1}{40} x^5 + ....\\
		\tanh x &= x - \frac{1}{3} x^3 + \frac{2}{15} x^5 - \frac{17}{315} x^7 + \frac{62}{2835} x^9 + ....\\
		\tanh^{-1} x &= x + \frac{1}{3} x^3 + \frac{1}{5} x^5 + \frac{1}{7} x^7 + \frac{1}{9} x^{9} + ....\\
		\end{align}


\end{document}