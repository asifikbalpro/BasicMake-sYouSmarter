%----------------------------------------------------------------------------------------
%	CHAPTER 4
%----------------------------------------------------------------------------------------

\chapterimage{head1.png} % Chapter heading image

\chapter{Complex Number}

\section{Complex Number}
	1) Imaginary number : square of imaginary number is -1 or negative.
	\begin{align}
		i = \sqrt{-1} \\
		(i)^2 = (\sqrt{-1})^2(\sqrt{-1})^2 \\
		= (-1)^\frac{1}{2} (-1)^\frac{1}{2} \\
		= (-1)^{\frac{1}{2}+\frac{1}{2}} \\
		= (-1)^1 \\
		= -1 \\
		\\
		large \ imaginary\ number\ \sqrt{-55} = \sqrt{55}i \\
		=\sqrt{-55} \\
		=\sqrt{ss * (-1)} \\
		=\sqrt{55}\sqrt{(-1)} \\
		=\sqrt{55}i
	\end{align}
	2.
	\begin{align}
		complex \ number \\
		z = a + ib \\
		\textbf{where a and b are real number.} \\
		a = Re(z) \\
		b = Im(z) \\
	\end{align}
	3.

	
	
	
	
	
	
\subsection{Euler formula}
	1.
	\begin{align}
		e^{i\theta} = \cos\theta+i\sin \theta \\
		z = a + ib \\
		given, \ a + ib = ||z||=\sqrt{a^2+b^2} \\
		||z|| = \sqrt{a^2+b^2}	 \\
		\cos \theta = \frac{Re(z)}{||z||} = \frac{a}{\sqrt{a^2+b^2}} \\
		\sin\theta = \frac{Im(z)}{||z||} = \frac{b }{\sqrt{a^2+b^2}} \\
		z = ||z|| e^{i\theta} \\
		= \sqrt{a^2+b^2}(\cos\theta+i\sin\theta) \\
		= \sqrt{a^2+b^2}(\frac{a}{\sqrt{a^2+b^2}}+i \frac{b}{\sqrt{a^2+b^2}}) \\
		= a + ib \\
 	\end{align}
 	
 	
 	
 	
\subsection{Euler Rotation}



\subsection{Euler Rotation (Rotation coordinate transformation )}
 	
 	
 \section{Spring}
 
 	
\subsection{Mass spring system}
	mass m,
	distance form equilibrium x,
	spring constant k,
	Hook's law  F=-kx
	force is proportional to the displacement
	\begin{align}
		unit [F] = \ newton \ = MLT^{-2} \\
		[x] = \ meter \ L \\
		[x] = \frac{MLT^{-2}}{L} = ML^{-2} \\
	\end{align}
	note:
	\begin{align}
		Notation: \\
		\frac{dx}{dt} = x \\
		\frac{d^2x}{dt^2} = x = \frac{d}{dt}(\frac{dx}{dt}) = \frac{dv}{dt} = a \\
	\end{align}
	Newton's law, F = ma = mx %there is a 2 notetion on x on xm.
	since no offer force is action on the spring xm = -kx  %there is a 2 notetion on xm on x.
	this is the equation of notion (EOM) of a mass spring system.
	\begin{align}
		Solution o EOM \\
		mx^{..} = -kx \\
		=> x^{..} = -\frac{k}{m}x \\
		=> \frac{d^2x}{dt^2} = -\omega^2x \\
		Assume \  the \ Solution \ is, \ x(t) = Ae^{i\omega t}+ Be^{-i\omega t} = x_1(t)+x_2(t) \ trial. \\
		\\
	\end{align}
	\begin{align}
		Checking \ for \ x_1(t) \\
		L.H.S,
		\frac{d^2x(t)}{dt^2} \\
		= A	\frac{d^2}{dt^2}e^{i\omega t} \\
		= A \frac{d}{dt}(\frac{de^{i\omega t}}{dt}(i\omega)) \\
		= A(i\omega) \frac{d}{dt}e^{i\omega t }\\
		= A(i \omega) e^{i\omega t}(i\omega) \\
		= A(i\omega)^2 e^{i\omega t} \\
		= i^2\omega^2 Ae^{i\omega t} \\
		= -\omega^2 x_1(t) \ \ \ R.H.S
	\end{align}
	\begin{align}
		Checking for x_2(t) \\
		L.H.S, \\
		\frac{d^2}{dt^2}x_2	(t)	 \\
		= (-i \omega)^2 Be^{i\omega t} \\
		= (-i \omega)^2 x_2(t) \\
		= (-i\omega) x_2(t) \\
	\end{align}
	constant A and B determined form boundary condition.
	\begin{align}
		say, \ x(t) = Be^{-i\omega t} \\
		= B[\cos(\omega t)+ i \sin(\omega t)] \\
	\end{align}
	
	
\subsection{Energy stoned in the spring}



\subsection{Frequency}






	









